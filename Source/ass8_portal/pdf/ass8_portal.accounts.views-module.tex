%
% API Documentation for API Documentation
% Module ass8_portal.accounts.views
%
% Generated by epydoc 3.0
% [Mon May 25 11:06:42 2009]
%

%%%%%%%%%%%%%%%%%%%%%%%%%%%%%%%%%%%%%%%%%%%%%%%%%%%%%%%%%%%%%%%%%%%%%%%%%%%
%%                          Module Description                           %%
%%%%%%%%%%%%%%%%%%%%%%%%%%%%%%%%%%%%%%%%%%%%%%%%%%%%%%%%%%%%%%%%%%%%%%%%%%%

    \index{ass8\_portal \textit{(package)}!ass8\_portal.accounts \textit{(package)}!ass8\_portal.accounts.views \textit{(module)}|(}
\section{Module ass8\_portal.accounts.views}

    \label{ass8_portal:accounts:views}
Plik odpowiedzialny za definicje widoków aplikacji 'accounts'. Aplikacja ta
z kolei odpowiada za funkcjonalość związaną z zarządzaniem kontem 
użytkownika.

\begin{itemize}
\setlength{\parskip}{0.6ex}
  \item MESSAGE\_CODES - jest to lista przypisująca kodom 0, 1, 2 odpowiednie 
    wiadomości o błędzie

\end{itemize}


%%%%%%%%%%%%%%%%%%%%%%%%%%%%%%%%%%%%%%%%%%%%%%%%%%%%%%%%%%%%%%%%%%%%%%%%%%%
%%                               Functions                               %%
%%%%%%%%%%%%%%%%%%%%%%%%%%%%%%%%%%%%%%%%%%%%%%%%%%%%%%%%%%%%%%%%%%%%%%%%%%%

  \subsection{Functions}

    \label{ass8_portal:accounts:views:user_login}
    \index{ass8\_portal \textit{(package)}!ass8\_portal.accounts \textit{(package)}!ass8\_portal.accounts.views \textit{(module)}!ass8\_portal.accounts.views.user\_login \textit{(function)}}

    \vspace{0.5ex}

\hspace{.8\funcindent}\begin{boxedminipage}{\funcwidth}

    \raggedright \textbf{user\_login}(\textit{request})

    \vspace{-1.5ex}

    \rule{\textwidth}{0.5\fboxrule}
\setlength{\parskip}{2ex}
    Widok odpowiedzialny za logowanie użytkownika. Sprawdzane jest czy 
    użytkownik nie jest juz zalogowany, jeśli nie to sprawdzane jest jego 
    login i hasło. W przypadku poprawnych danych użytkownik jest 
    przekierowywany na stronę swojego profilu, natomiast w przypadku błędu 
    wyświetlany jest odpowiedni komunikat.

\setlength{\parskip}{1ex}
      \textbf{Parameters}
      \vspace{-1ex}

      \begin{quote}
        \begin{Ventry}{xxxxxxx}

          \item[request]

          żądanie przeglądarki

        \end{Ventry}

      \end{quote}

    \end{boxedminipage}

    \label{ass8_portal:accounts:views:user_logout}
    \index{ass8\_portal \textit{(package)}!ass8\_portal.accounts \textit{(package)}!ass8\_portal.accounts.views \textit{(module)}!ass8\_portal.accounts.views.user\_logout \textit{(function)}}

    \vspace{0.5ex}

\hspace{.8\funcindent}\begin{boxedminipage}{\funcwidth}

    \raggedright \textbf{user\_logout}(\textit{request})

    \vspace{-1.5ex}

    \rule{\textwidth}{0.5\fboxrule}
\setlength{\parskip}{2ex}
    Widok odpowiedzialny za wylogowanie użytkownika. Sprawdzane jest czy 
    użytkownik jest zalogowany, jeśli tak następuje jego wylogowanie i 
    przekierowanie na stronę główną.

\setlength{\parskip}{1ex}
      \textbf{Parameters}
      \vspace{-1ex}

      \begin{quote}
        \begin{Ventry}{xxxxxxx}

          \item[request]

          żądanie przeglądarki

        \end{Ventry}

      \end{quote}

    \end{boxedminipage}

    \label{ass8_portal:accounts:views:index}
    \index{ass8\_portal \textit{(package)}!ass8\_portal.accounts \textit{(package)}!ass8\_portal.accounts.views \textit{(module)}!ass8\_portal.accounts.views.index \textit{(function)}}

    \vspace{0.5ex}

\hspace{.8\funcindent}\begin{boxedminipage}{\funcwidth}

    \raggedright \textbf{index}(\textit{request})

    \vspace{-1.5ex}

    \rule{\textwidth}{0.5\fboxrule}
\setlength{\parskip}{2ex}
    Widok odpowiedzialny za wyświetlanie głównej strony portalu. 
    Przekazywana jest informacja o tym czy użytkownik przeglądający stronę 
    jest zalogowany czy nie.

\setlength{\parskip}{1ex}
      \textbf{Parameters}
      \vspace{-1ex}

      \begin{quote}
        \begin{Ventry}{xxxxxxx}

          \item[request]

          żądanie przeglądarki

        \end{Ventry}

      \end{quote}

    \end{boxedminipage}

    \label{ass8_portal:accounts:views:register}
    \index{ass8\_portal \textit{(package)}!ass8\_portal.accounts \textit{(package)}!ass8\_portal.accounts.views \textit{(module)}!ass8\_portal.accounts.views.register \textit{(function)}}

    \vspace{0.5ex}

\hspace{.8\funcindent}\begin{boxedminipage}{\funcwidth}

    \raggedright \textbf{register}(\textit{request})

    \vspace{-1.5ex}

    \rule{\textwidth}{0.5\fboxrule}
\setlength{\parskip}{2ex}
    Widok odpowiedzialny za rejestrację nowego użytkownika. Generowany jest
    formularz a przy próbie wysłania sprawdzane są podane w nim dane. 
    Najważniejszą z nich jest sprawdzenie czy podany przez użytkownika 
    login jest unikalny.

\setlength{\parskip}{1ex}
      \textbf{Parameters}
      \vspace{-1ex}

      \begin{quote}
        \begin{Ventry}{xxxxxxx}

          \item[request]

          żądanie przeglądarki

        \end{Ventry}

      \end{quote}

    \end{boxedminipage}

    \label{ass8_portal:accounts:views:profile_view}
    \index{ass8\_portal \textit{(package)}!ass8\_portal.accounts \textit{(package)}!ass8\_portal.accounts.views \textit{(module)}!ass8\_portal.accounts.views.profile\_view \textit{(function)}}

    \vspace{0.5ex}

\hspace{.8\funcindent}\begin{boxedminipage}{\funcwidth}

    \raggedright \textbf{profile\_view}(\textit{request}, \textit{username})

    \vspace{-1.5ex}

    \rule{\textwidth}{0.5\fboxrule}
\setlength{\parskip}{2ex}
    Widok odpowiedzialny za wyświetlanie profliu użytkownika. Aby funkcja 
    została wywołana użytkownik musi być zalogowany. Jeśli nie jest - 
    zostanie przekierowany na stronę logowania, a po poprawnym logowaniu na
    stronę profilu użytkownika którą chciał obejrzeć wcześniej. W sytuacji 
    gdy użytkownik chce obejrzeć swój profil informacja ta jest odpowienio 
    interpretowana i wpływa na wygląd strony.

\setlength{\parskip}{1ex}
      \textbf{Parameters}
      \vspace{-1ex}

      \begin{quote}
        \begin{Ventry}{xxxxxxxx}

          \item[request]

          żądanie przeglądarki

          \item[username]

          login użytkownika którego profil chcemy obejrzeć

        \end{Ventry}

      \end{quote}

    \end{boxedminipage}

    \label{ass8_portal:accounts:views:latest_users}
    \index{ass8\_portal \textit{(package)}!ass8\_portal.accounts \textit{(package)}!ass8\_portal.accounts.views \textit{(module)}!ass8\_portal.accounts.views.latest\_users \textit{(function)}}

    \vspace{0.5ex}

\hspace{.8\funcindent}\begin{boxedminipage}{\funcwidth}

    \raggedright \textbf{latest\_users}(\textit{request})

    \vspace{-1.5ex}

    \rule{\textwidth}{0.5\fboxrule}
\setlength{\parskip}{2ex}
    Widok odpowiedzialny za wyświetlanie listy ostatnio zarejestrowanych 
    użytkowników.

\setlength{\parskip}{1ex}
      \textbf{Parameters}
      \vspace{-1ex}

      \begin{quote}
        \begin{Ventry}{xxxxxxx}

          \item[request]

          żądanie przeglądarki

        \end{Ventry}

      \end{quote}

    \end{boxedminipage}

    \label{ass8_portal:accounts:views:search}
    \index{ass8\_portal \textit{(package)}!ass8\_portal.accounts \textit{(package)}!ass8\_portal.accounts.views \textit{(module)}!ass8\_portal.accounts.views.search \textit{(function)}}

    \vspace{0.5ex}

\hspace{.8\funcindent}\begin{boxedminipage}{\funcwidth}

    \raggedright \textbf{search}(\textit{request})

    \vspace{-1.5ex}

    \rule{\textwidth}{0.5\fboxrule}
\setlength{\parskip}{2ex}
    Widok odpowiedzialny za wyszukiwanie użytkowników. Aby funkcja została 
    wywołana użytkownik musi być zalogowany. Jeśli nie jest - zostanie 
    przekierowany na stronę logowania, a po poprawnym logowaniu ponownie na
    stronę wyszukiwarki. Wyszukiwanie odbywa się poprzez porównanie 
    porównanie tekstu wprowadzonego przez użytkownika z loginami 
    użytkowników w bazie.

\setlength{\parskip}{1ex}
      \textbf{Parameters}
      \vspace{-1ex}

      \begin{quote}
        \begin{Ventry}{xxxxxxx}

          \item[request]

          żądanie przeglądarki

        \end{Ventry}

      \end{quote}

    \end{boxedminipage}

    \label{ass8_portal:accounts:views:profile_edit}
    \index{ass8\_portal \textit{(package)}!ass8\_portal.accounts \textit{(package)}!ass8\_portal.accounts.views \textit{(module)}!ass8\_portal.accounts.views.profile\_edit \textit{(function)}}

    \vspace{0.5ex}

\hspace{.8\funcindent}\begin{boxedminipage}{\funcwidth}

    \raggedright \textbf{profile\_edit}(\textit{request}, \textit{username})

    \vspace{-1.5ex}

    \rule{\textwidth}{0.5\fboxrule}
\setlength{\parskip}{2ex}
    Widok odpowiedzialny za edycję profliu użytkownika. Aby funkcja została
    wywołana użytkownik musi być zalogowany. Jeśli nie jest - zostanie 
    przekierowany na stronę logowania, a po poprawnym ponownie na stronę 
    edycji. Sprawdzane jest czy użytkownik chce zmieniać swój profil i 
    jeśli nie to wyświetlana jest informacja o błędzie.

\setlength{\parskip}{1ex}
      \textbf{Parameters}
      \vspace{-1ex}

      \begin{quote}
        \begin{Ventry}{xxxxxxxx}

          \item[request]

          żądanie przeglądarki

          \item[username]

          login użytkownika którego profil chcemy zmieniać

        \end{Ventry}

      \end{quote}

    \end{boxedminipage}

    \label{ass8_portal:accounts:views:profile_save}
    \index{ass8\_portal \textit{(package)}!ass8\_portal.accounts \textit{(package)}!ass8\_portal.accounts.views \textit{(module)}!ass8\_portal.accounts.views.profile\_save \textit{(function)}}

    \vspace{0.5ex}

\hspace{.8\funcindent}\begin{boxedminipage}{\funcwidth}

    \raggedright \textbf{profile\_save}(\textit{request}, \textit{username})

    \vspace{-1.5ex}

    \rule{\textwidth}{0.5\fboxrule}
\setlength{\parskip}{2ex}
    Widok odpowiedzialny za zapis zmian w profliu użytkownika. Aby funkcja 
    została wywołana użytkownik musi być zalogowany. Jeśli nie jest - 
    zostanie przekierowany na stronę logowania, a po poprawnym ponownie 
    podjęta zostanie próba zapisu. Sprawdzane jest czy użytkownik chce 
    zpisać swój profil i jeśli nie to wyświetlana jest informacja o 
    błędzie.

\setlength{\parskip}{1ex}
      \textbf{Parameters}
      \vspace{-1ex}

      \begin{quote}
        \begin{Ventry}{xxxxxxxx}

          \item[request]

          żądanie przeglądarki

          \item[username]

          login użytkownika którego profil chcemy zapisać

        \end{Ventry}

      \end{quote}

    \end{boxedminipage}

    \label{ass8_portal:accounts:views:profile_delete}
    \index{ass8\_portal \textit{(package)}!ass8\_portal.accounts \textit{(package)}!ass8\_portal.accounts.views \textit{(module)}!ass8\_portal.accounts.views.profile\_delete \textit{(function)}}

    \vspace{0.5ex}

\hspace{.8\funcindent}\begin{boxedminipage}{\funcwidth}

    \raggedright \textbf{profile\_delete}(\textit{request}, \textit{username})

    \vspace{-1.5ex}

    \rule{\textwidth}{0.5\fboxrule}
\setlength{\parskip}{2ex}
    Widok odpowiedzialny za kasowanie profliu użytkownika. Aby funkcja 
    została wywołana użytkownik musi być zalogowany. Jeśli nie jest - 
    zostanie przekierowany na stronę logowania, a po poprawnym logowaniu 
    ponownie wykonywana jest próba kasowania profilu. Sprawdzane jest czy 
    użytkownik chce sksasować swój profil, jeśli nie wyświetlany jest 
    komunikat o błędzie. Kasowanie profilu usuwa poza profilem wsyzstkie 
    pliki i znajomości użytkownika z bazy.

\setlength{\parskip}{1ex}
      \textbf{Parameters}
      \vspace{-1ex}

      \begin{quote}
        \begin{Ventry}{xxxxxxxx}

          \item[request]

          żądanie przeglądarki

          \item[username]

          login użytkownika którego profil chcemy skasować

        \end{Ventry}

      \end{quote}

    \end{boxedminipage}


%%%%%%%%%%%%%%%%%%%%%%%%%%%%%%%%%%%%%%%%%%%%%%%%%%%%%%%%%%%%%%%%%%%%%%%%%%%
%%                               Variables                               %%
%%%%%%%%%%%%%%%%%%%%%%%%%%%%%%%%%%%%%%%%%%%%%%%%%%%%%%%%%%%%%%%%%%%%%%%%%%%

  \subsection{Variables}

    \vspace{-1cm}
\hspace{\varindent}\begin{longtable}{|p{\varnamewidth}|p{\vardescrwidth}|l}
\cline{1-2}
\cline{1-2} \centering \textbf{Name} & \centering \textbf{Description}& \\
\cline{1-2}
\endhead\cline{1-2}\multicolumn{3}{r}{\small\textit{continued on next page}}\\\endfoot\cline{1-2}
\endlastfoot\raggedright M\-E\-S\-S\-A\-G\-E\-\_\-C\-O\-D\-E\-S\- & \raggedright \textbf{Value:} 
{\tt 'Warning', 'Information', 'Error'}&\\
\cline{1-2}
\end{longtable}


%%%%%%%%%%%%%%%%%%%%%%%%%%%%%%%%%%%%%%%%%%%%%%%%%%%%%%%%%%%%%%%%%%%%%%%%%%%
%%                           Class Description                           %%
%%%%%%%%%%%%%%%%%%%%%%%%%%%%%%%%%%%%%%%%%%%%%%%%%%%%%%%%%%%%%%%%%%%%%%%%%%%

    \index{ass8\_portal \textit{(package)}!ass8\_portal.accounts \textit{(package)}!ass8\_portal.accounts.views \textit{(module)}!ass8\_portal.accounts.views.Message \textit{(class)}|(}
\subsection{Class Message}

    \label{ass8_portal:accounts:views:Message}
\begin{tabular}{cccccc}
% Line for object, linespec=[False]
\multicolumn{2}{r}{\settowidth{\BCL}{object}\multirow{2}{\BCL}{object}}
&&
  \\\cline{3-3}
  &&\multicolumn{1}{c|}{}
&&
  \\
&&\multicolumn{2}{l}{\textbf{ass8\_portal.accounts.views.Message}}
\end{tabular}

Klasa która odpowiedzialna jest za przechowywanie typu i treści komunikatów
wysylanych do użytkownika. Typem jest wartość z listy MESSAGE\_CODES, 
treścią dowolny komunikat który chcemy wyświeltlić. W zależności od 
podanego typu komunikatu wiadomość zostanie wyświetlona w innym kolorze, 
aby na pierwszy rzut oka użytkownik wiedział czy dana informacja jest 
błędem czy potwierdzeniem wykonania danej operacji.


%%%%%%%%%%%%%%%%%%%%%%%%%%%%%%%%%%%%%%%%%%%%%%%%%%%%%%%%%%%%%%%%%%%%%%%%%%%
%%                                Methods                                %%
%%%%%%%%%%%%%%%%%%%%%%%%%%%%%%%%%%%%%%%%%%%%%%%%%%%%%%%%%%%%%%%%%%%%%%%%%%%

  \subsubsection{Methods}

    \vspace{0.5ex}

\hspace{.8\funcindent}\begin{boxedminipage}{\funcwidth}

    \raggedright \textbf{\_\_init\_\_}(\textit{self}, \textit{type}, \textit{content})

    \vspace{-1.5ex}

    \rule{\textwidth}{0.5\fboxrule}
\setlength{\parskip}{2ex}
    Konstruktor wiadomości.

\setlength{\parskip}{1ex}
      \textbf{Parameters}
      \vspace{-1ex}

      \begin{quote}
        \begin{Ventry}{xxxxxxx}

          \item[type]

          typ wiadomości - liczba z zakresu 0 - 2

          \item[content]

          treść wiadomości którą chcemy wyświetlić

        \end{Ventry}

      \end{quote}

      Overrides: object.\_\_init\_\_

    \end{boxedminipage}

    \label{ass8_portal:accounts:views:Message:__unicode__}
    \index{ass8\_portal \textit{(package)}!ass8\_portal.accounts \textit{(package)}!ass8\_portal.accounts.views \textit{(module)}!ass8\_portal.accounts.views.Message \textit{(class)}!ass8\_portal.accounts.views.Message.\_\_unicode\_\_ \textit{(method)}}

    \vspace{0.5ex}

\hspace{.8\funcindent}\begin{boxedminipage}{\funcwidth}

    \raggedright \textbf{\_\_unicode\_\_}(\textit{self})

    \vspace{-1.5ex}

    \rule{\textwidth}{0.5\fboxrule}
\setlength{\parskip}{2ex}
    Metoda odpowiedzialna za ewentualne wypisanie wiadomości w postaci 
    innej niż HTML(np. na ekran terminala).

\setlength{\parskip}{1ex}
    \end{boxedminipage}


\large{\textbf{\textit{Inherited from object}}}

\begin{quote}
\_\_delattr\_\_(), \_\_format\_\_(), \_\_getattribute\_\_(), \_\_hash\_\_(), \_\_new\_\_(), \_\_reduce\_\_(), \_\_reduce\_ex\_\_(), \_\_repr\_\_(), \_\_setattr\_\_(), \_\_sizeof\_\_(), \_\_str\_\_(), \_\_subclasshook\_\_()
\end{quote}

%%%%%%%%%%%%%%%%%%%%%%%%%%%%%%%%%%%%%%%%%%%%%%%%%%%%%%%%%%%%%%%%%%%%%%%%%%%
%%                              Properties                               %%
%%%%%%%%%%%%%%%%%%%%%%%%%%%%%%%%%%%%%%%%%%%%%%%%%%%%%%%%%%%%%%%%%%%%%%%%%%%

  \subsubsection{Properties}

    \vspace{-1cm}
\hspace{\varindent}\begin{longtable}{|p{\varnamewidth}|p{\vardescrwidth}|l}
\cline{1-2}
\cline{1-2} \centering \textbf{Name} & \centering \textbf{Description}& \\
\cline{1-2}
\endhead\cline{1-2}\multicolumn{3}{r}{\small\textit{continued on next page}}\\\endfoot\cline{1-2}
\endlastfoot\multicolumn{2}{|l|}{\textit{Inherited from object}}\\
\multicolumn{2}{|p{\varwidth}|}{\raggedright \_\_class\_\_}\\
\cline{1-2}
\end{longtable}

    \index{ass8\_portal \textit{(package)}!ass8\_portal.accounts \textit{(package)}!ass8\_portal.accounts.views \textit{(module)}!ass8\_portal.accounts.views.Message \textit{(class)}|)}
    \index{ass8\_portal \textit{(package)}!ass8\_portal.accounts \textit{(package)}!ass8\_portal.accounts.views \textit{(module)}|)}
