%
% API Documentation for API Documentation
% Module ass8_portal.friends.views
%
% Generated by epydoc 3.0
% [Mon May 25 11:06:42 2009]
%

%%%%%%%%%%%%%%%%%%%%%%%%%%%%%%%%%%%%%%%%%%%%%%%%%%%%%%%%%%%%%%%%%%%%%%%%%%%
%%                          Module Description                           %%
%%%%%%%%%%%%%%%%%%%%%%%%%%%%%%%%%%%%%%%%%%%%%%%%%%%%%%%%%%%%%%%%%%%%%%%%%%%

    \index{ass8\_portal \textit{(package)}!ass8\_portal.friends \textit{(package)}!ass8\_portal.friends.views \textit{(module)}|(}
\section{Module ass8\_portal.friends.views}

    \label{ass8_portal:friends:views}
Plik odpowiedzialny za definicje widoków aplikacji 'files'. Aplikacja ta z 
kolei odpowiada za funkcjonalość związaną z zarządzaniem plikami 
użytkownika.

\begin{itemize}
\setlength{\parskip}{0.6ex}
  \item FRIEND\_FUNCTION\_MAP - to słownik zawierający funkcje pobierające 
    odpowienie typy znajomości.

\end{itemize}


%%%%%%%%%%%%%%%%%%%%%%%%%%%%%%%%%%%%%%%%%%%%%%%%%%%%%%%%%%%%%%%%%%%%%%%%%%%
%%                               Functions                               %%
%%%%%%%%%%%%%%%%%%%%%%%%%%%%%%%%%%%%%%%%%%%%%%%%%%%%%%%%%%%%%%%%%%%%%%%%%%%

  \subsection{Functions}

    \label{ass8_portal:friends:views:friends_manage}
    \index{ass8\_portal \textit{(package)}!ass8\_portal.friends \textit{(package)}!ass8\_portal.friends.views \textit{(module)}!ass8\_portal.friends.views.friends\_manage \textit{(function)}}

    \vspace{0.5ex}

\hspace{.8\funcindent}\begin{boxedminipage}{\funcwidth}

    \raggedright \textbf{friends\_manage}(\textit{request}, \textit{username})

    \vspace{-1.5ex}

    \rule{\textwidth}{0.5\fboxrule}
\setlength{\parskip}{2ex}
    Widok odpowiedzialny za zarządzanie listą znajomych użytkownika. Aby 
    funkcja została wywołana użytkownik musi być zalogowany. Jeśli nie jest
    - zostanie przekierowany na stronę logowania, a po poprawnym logowaniu 
    ponownie na stronę zarządzającą listą.

\setlength{\parskip}{1ex}
      \textbf{Parameters}
      \vspace{-1ex}

      \begin{quote}
        \begin{Ventry}{xxxxxxxx}

          \item[request]

          żądanie przeglądarki

          \item[username]

          login użytkownika którego listą chcemy zarządzać

        \end{Ventry}

      \end{quote}

    \end{boxedminipage}

    \label{ass8_portal:friends:views:friend_list}
    \index{ass8\_portal \textit{(package)}!ass8\_portal.friends \textit{(package)}!ass8\_portal.friends.views \textit{(module)}!ass8\_portal.friends.views.friend\_list \textit{(function)}}

    \vspace{0.5ex}

\hspace{.8\funcindent}\begin{boxedminipage}{\funcwidth}

    \raggedright \textbf{friend\_list}(\textit{request}, \textit{list\_type}, \textit{username})

    \vspace{-1.5ex}

    \rule{\textwidth}{0.5\fboxrule}
\setlength{\parskip}{2ex}
    Widok odpowiedzialny za wyświetlenie znajomości danego typu. Aby 
    funkcja została wywołana użytkownik musi być zalogowany. Jeśli nie jest
    - zostanie przekierowany na stronę logowania, a po poprawnym logowaniu 
    ponownie na stronę wyświetlającą listę.

\setlength{\parskip}{1ex}
      \textbf{Parameters}
      \vspace{-1ex}

      \begin{quote}
        \begin{Ventry}{xxxxxxxxx}

          \item[request]

          żądanie przeglądarki

          \item[list\_type]

          typ znajomości które chcemy wyświetlić

          \item[username]

          login użytkownika którego listą chcemy zarządzać

        \end{Ventry}

      \end{quote}

    \end{boxedminipage}

    \label{ass8_portal:friends:views:add_friend}
    \index{ass8\_portal \textit{(package)}!ass8\_portal.friends \textit{(package)}!ass8\_portal.friends.views \textit{(module)}!ass8\_portal.friends.views.add\_friend \textit{(function)}}

    \vspace{0.5ex}

\hspace{.8\funcindent}\begin{boxedminipage}{\funcwidth}

    \raggedright \textbf{add\_friend}(\textit{request}, \textit{username})

    \vspace{-1.5ex}

    \rule{\textwidth}{0.5\fboxrule}
\setlength{\parskip}{2ex}
    Widok odpowiedzialny za dodanie znajomego do listy. Aby funkcja została
    wywołana użytkownik musi być zalogowany. Jeśli nie jest - zostanie 
    przekierowany na stronę logowania, a po poprawnym logowaniu ponownie na
    stronę wyświetlającą listę. Sprawdzane jest czy dany użytkownik jest 
    już dodany do listy i jeśli tak to wyświetlany jest odpowiedni 
    komunikat. Również w przypadku próby dodania siebie do znajomych 
    wyświetlany jest komunikat o błędzie.

\setlength{\parskip}{1ex}
      \textbf{Parameters}
      \vspace{-1ex}

      \begin{quote}
        \begin{Ventry}{xxxxxxxx}

          \item[request]

          żądanie przeglądarki

          \item[username]

          login użytkownika którego listą chcemy zarządzać

        \end{Ventry}

      \end{quote}

    \end{boxedminipage}

    \label{ass8_portal:friends:views:del_friend}
    \index{ass8\_portal \textit{(package)}!ass8\_portal.friends \textit{(package)}!ass8\_portal.friends.views \textit{(module)}!ass8\_portal.friends.views.del\_friend \textit{(function)}}

    \vspace{0.5ex}

\hspace{.8\funcindent}\begin{boxedminipage}{\funcwidth}

    \raggedright \textbf{del\_friend}(\textit{request}, \textit{username})

    \vspace{-1.5ex}

    \rule{\textwidth}{0.5\fboxrule}
\setlength{\parskip}{2ex}
    Widok odpowiedzialny za usnięcie znajomego z listy. Aby funkcja została
    wywołana użytkownik musi być zalogowany. Jeśli nie jest - zostanie 
    przekierowany na stronę logowania, a po poprawnym logowaniu ponownie na
    stronę wyświetlającą listę. Sprawdzan jest czy dany użytkownik jest 
    dodany do listy i jeśli nie to wyświetlany jest odpowiedni komunikat.

\setlength{\parskip}{1ex}
      \textbf{Parameters}
      \vspace{-1ex}

      \begin{quote}
        \begin{Ventry}{xxxxxxxx}

          \item[request]

          żądanie przeglądarki

          \item[username]

          login użytkownika którego listą chcemy zarządzać

        \end{Ventry}

      \end{quote}

    \end{boxedminipage}


%%%%%%%%%%%%%%%%%%%%%%%%%%%%%%%%%%%%%%%%%%%%%%%%%%%%%%%%%%%%%%%%%%%%%%%%%%%
%%                               Variables                               %%
%%%%%%%%%%%%%%%%%%%%%%%%%%%%%%%%%%%%%%%%%%%%%%%%%%%%%%%%%%%%%%%%%%%%%%%%%%%

  \subsection{Variables}

    \vspace{-1cm}
\hspace{\varindent}\begin{longtable}{|p{\varnamewidth}|p{\vardescrwidth}|l}
\cline{1-2}
\cline{1-2} \centering \textbf{Name} & \centering \textbf{Description}& \\
\cline{1-2}
\endhead\cline{1-2}\multicolumn{3}{r}{\small\textit{continued on next page}}\\\endfoot\cline{1-2}
\endlastfoot\raggedright F\-R\-I\-E\-N\-D\-\_\-F\-U\-N\-C\-T\-I\-O\-N\-\_\-M\-A\-P\- & \raggedright \textbf{Value:} 
{\tt \{'followers': get\_my\_followers, 'following': get\_my\_follo\texttt{...}}&\\
\cline{1-2}
\end{longtable}

    \index{ass8\_portal \textit{(package)}!ass8\_portal.friends \textit{(package)}!ass8\_portal.friends.views \textit{(module)}|)}
