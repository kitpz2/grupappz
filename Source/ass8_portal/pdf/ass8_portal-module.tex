%
% API Documentation for API Documentation
% Package ass8_portal
%
% Generated by epydoc 3.0
% [Mon May 25 11:06:42 2009]
%

%%%%%%%%%%%%%%%%%%%%%%%%%%%%%%%%%%%%%%%%%%%%%%%%%%%%%%%%%%%%%%%%%%%%%%%%%%%
%%                          Module Description                           %%
%%%%%%%%%%%%%%%%%%%%%%%%%%%%%%%%%%%%%%%%%%%%%%%%%%%%%%%%%%%%%%%%%%%%%%%%%%%

    \index{ass8\_portal \textit{(package)}|(}
\section{Package ass8\_portal}

    \label{ass8_portal}
ASS.8 - Portal to aplikacja internetowa oparta o framework django. 
Framework ten spełnia nieco zmodyfikowany paradygmat tworzenia aplikacji 
MVC Model View Controler. W przypadku django możemy mówić o MTV - Model 
Template View. Gdzie:

\begin{itemize}
\setlength{\parskip}{0.6ex}
  \item Model - to część aplikacji odpowiedzilna za modelowanie danych w bazie.
    Praca z modelami nie wymaga znajomości SQL, a jedynie wewnętrznych 
    funkcji frameworka.

  \item Template - to szablon strony którą widzi użytkowniki. System szablonów 
    poza znacznikami HTML udostępnia też wewnętrzne znaczniki która 
    udostępniają prostą nie liniowość w postacji warunków czy pętli.

  \item View - to ta część aplikacji która odpowiada za operacje na bazie 
    danych, ich modyfikacje i przekazanie w odpowieniej formie do szablonu.

\end{itemize}

ASS.8 składa się z djangowego projektu który scala w sobie 3 
współdziałające aplikacje:

\begin{itemize}
\setlength{\parskip}{0.6ex}
  \item accounts - aplikacja odpowiedzialna za zarządzanie kontami użytkownika

  \item files - aplikacja odpowiedzialna za zarządzanie plikami użytkownika

  \item friends - aplikacja odpowiedzialna za zarządzanie znajomościami 
    użytkownika

\end{itemize}


%%%%%%%%%%%%%%%%%%%%%%%%%%%%%%%%%%%%%%%%%%%%%%%%%%%%%%%%%%%%%%%%%%%%%%%%%%%
%%                                Modules                                %%
%%%%%%%%%%%%%%%%%%%%%%%%%%%%%%%%%%%%%%%%%%%%%%%%%%%%%%%%%%%%%%%%%%%%%%%%%%%

\subsection{Modules}

\begin{itemize}
\setlength{\parskip}{0ex}
\item \textbf{accounts}: Moduł 'accounts' odpowiada za funkcjonalość związaną z zarządzaniem kontem 
użytkownika.



  \textit{(Section \ref{ass8_portal:accounts}, p.~\pageref{ass8_portal:accounts})}

  \begin{itemize}
\setlength{\parskip}{0ex}
    \item \textbf{admin}: Plik rejestruje model Konto w panelu administracyjnym.



  \textit{(Section \ref{ass8_portal:accounts:admin}, p.~\pageref{ass8_portal:accounts:admin})}

    \item \textbf{forms}: Plik zawiera opis formularzy wykorzystywanych w aplikacji 'accounts'.



  \textit{(Section \ref{ass8_portal:accounts:forms}, p.~\pageref{ass8_portal:accounts:forms})}

    \item \textbf{models}: Plik zawiera opis modeli dla bazy danych dla aplikacji 'accounts'.



  \textit{(Section \ref{ass8_portal:accounts:models}, p.~\pageref{ass8_portal:accounts:models})}

    \item \textbf{urls}: 
Plik zawiera ustawienia adresów URL dla aplikacji 'accounts', 
które użytkownik może wpisywać i skojarzone z nimi widoki które zostaną wywołane przy 
danym adresie URL. 


  \textit{(Section \ref{ass8_portal:accounts:urls}, p.~\pageref{ass8_portal:accounts:urls})}

    \item \textbf{views}: Plik odpowiedzialny za definicje widoków aplikacji 'accounts'.



  \textit{(Section \ref{ass8_portal:accounts:views}, p.~\pageref{ass8_portal:accounts:views})}

  \end{itemize}
\item \textbf{files}: Moduł 'files' odpowiada za funkcjonalość związaną z zarządzaniem plikami 
użytkownika.



  \textit{(Section \ref{ass8_portal:files}, p.~\pageref{ass8_portal:files})}

  \begin{itemize}
\setlength{\parskip}{0ex}
    \item \textbf{admin}: Plik rejestruje model Plik w panelu administracyjnym.



  \textit{(Section \ref{ass8_portal:files:admin}, p.~\pageref{ass8_portal:files:admin})}

    \item \textbf{forms}
  \textit{(Section \ref{ass8_portal:files:forms}, p.~\pageref{ass8_portal:files:forms})}

    \item \textbf{models}: Plik zawiera opis modeli dla bazy danych dla aplikacji 'files'.



  \textit{(Section \ref{ass8_portal:files:models}, p.~\pageref{ass8_portal:files:models})}

    \item \textbf{urls}: 
Plik zawiera ustawienia adresów URL dla aplikacji 'files', 
które użytkownik może wpisywać i skojarzone z nimi widoki które zostaną wywołane przy 
danym adresie URL. 


  \textit{(Section \ref{ass8_portal:files:urls}, p.~\pageref{ass8_portal:files:urls})}

    \item \textbf{views}
  \textit{(Section \ref{ass8_portal:files:views}, p.~\pageref{ass8_portal:files:views})}

  \end{itemize}
\item \textbf{friends}: Moduł 'friends' odpowiada za funkcjonalość związaną z zarządzaniem 
znajomościami użytkownika.



  \textit{(Section \ref{ass8_portal:friends}, p.~\pageref{ass8_portal:friends})}

  \begin{itemize}
\setlength{\parskip}{0ex}
    \item \textbf{admin}: Plik rejestruje model UserLink w panelu administracyjnym.



  \textit{(Section \ref{ass8_portal:friends:admin}, p.~\pageref{ass8_portal:friends:admin})}

    \item \textbf{helpers}: Plik zawiera definicję funkcji pomocniczych.



  \textit{(Section \ref{ass8_portal:friends:helpers}, p.~\pageref{ass8_portal:friends:helpers})}

    \item \textbf{models}: Plik zawiera opis modeli dla bazy danych dla aplikacji 'files'.



  \textit{(Section \ref{ass8_portal:friends:models}, p.~\pageref{ass8_portal:friends:models})}

    \item \textbf{urls}: 
Plik zawiera ustawienia adresów URL dla aplikacji 'firends', 
które użytkownik może wpisywać i skojarzone z nimi widoki które zostaną wywołane przy 
danym adresie URL. 


  \textit{(Section \ref{ass8_portal:friends:urls}, p.~\pageref{ass8_portal:friends:urls})}

    \item \textbf{views}: Plik odpowiedzialny za definicje widoków aplikacji 'files'.



  \textit{(Section \ref{ass8_portal:friends:views}, p.~\pageref{ass8_portal:friends:views})}

  \end{itemize}
\item \textbf{manage}: Plik odpowiedzialny za zarządzanie projektem.



  \textit{(Section \ref{ass8_portal:manage}, p.~\pageref{ass8_portal:manage})}

\item \textbf{settings}: Plik zawiera globalne ustawienia całego projektu.



  \textit{(Section \ref{ass8_portal:settings}, p.~\pageref{ass8_portal:settings})}

\item \textbf{urls}: 
Plik zawiera globalne ustawienia adresów URL, które użytkownik może wpisywać i skojarzone z nimi
widoki które zostaną wywołane przy danym adresie URL. 


  \textit{(Section \ref{ass8_portal:urls}, p.~\pageref{ass8_portal:urls})}

\end{itemize}


%%%%%%%%%%%%%%%%%%%%%%%%%%%%%%%%%%%%%%%%%%%%%%%%%%%%%%%%%%%%%%%%%%%%%%%%%%%
%%                               Variables                               %%
%%%%%%%%%%%%%%%%%%%%%%%%%%%%%%%%%%%%%%%%%%%%%%%%%%%%%%%%%%%%%%%%%%%%%%%%%%%

  \subsection{Variables}

    \vspace{-1cm}
\hspace{\varindent}\begin{longtable}{|p{\varnamewidth}|p{\vardescrwidth}|l}
\cline{1-2}
\cline{1-2} \centering \textbf{Name} & \centering \textbf{Description}& \\
\cline{1-2}
\endhead\cline{1-2}\multicolumn{3}{r}{\small\textit{continued on next page}}\\\endfoot\cline{1-2}
\endlastfoot\raggedright \_\-\_\-p\-a\-c\-k\-a\-g\-e\-\_\-\_\- & \raggedright \textbf{Value:} 
{\tt None}&\\
\cline{1-2}
\end{longtable}

    \index{ass8\_portal \textit{(package)}|)}
